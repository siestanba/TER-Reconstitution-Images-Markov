\documentclass[a4paper, 12pt]{article}
\title{Rapport TER}
\author{Yves Appriou}
\date{\today}
\usepackage{amsmath}
\usepackage{amssymb}

\begin{document}

\maketitle

\tableofcontents.
\newpage.
\section{Algorithmes de restauration d'images}
\subsection[Algorithme de Gibbs]{Algorithme de Gibbs}
Maintenant nous allons voir plus en detail l'algorithme de Metropolis, developpé dans les années \textbf{1960} il consiste en : 
\begin{itemize}
\item Choisir un pixel $s_{ij} $ aléatoirement dans l'image.
\item On calcule l'energie locale $U_s(x_0=\lambda_i| \mathcal{V}_s), \forall \lambda_i \in \mathbb{E}  $ pour chacun des états possibles. On obtient donc le vecteur des energies locales : 
\[
  U(x_0) = \left(
          \begin{array}{ll}
            U_s(x_0=\lambda_1| \mathcal{V}_s) \\
            U_s(x_0=\lambda_2| \mathcal{V}_s) \\
            ...\\
            U_s(x_0=\lambda_k| \mathcal{V}_s) \\
          \end{array}
        \right)
\]
\item Ainsi a partir de cette mesure on obtient une réalisation de la loi de Gibbs : 
\[
  \mu = P(x_1 = \lambda) = \frac{1}{Z} \left(
          \begin{array}{ll}
            \exp(-U_s(x_1=\lambda_1| \mathcal{V}_s)) \\
            \exp(-U_s(x_1=\lambda_2| \mathcal{V}_s)) \\
            ...\\
            \exp(-U_s(x_1=\lambda_k| \mathcal{V}_s)) \\
          \end{array}
        \right)
        , Z= \sum_{i\in \mathbb{E}} {U_s(x_1=i | \mathcal{V}_s)}
\]
La probabilité que le site $s_{ij}$ prenne la valeur $\lambda_i$ au temps n+1 est donnée par le $i^{\grave{e}me}$ élèment du vecteur de la loi de Gibbs.
\item Enfin on tire sur $\mathbb{E}$ muni de la loi $\mu$, et on remplace par l'état tiré
\end{itemize}
Pour le modele d'Ising par exemple on a $\mathbb{E} =\{0,1\}$;

\subsection[Algorithme de Metropolis]{Algorithme de Metropolis}


\end{document}
